\documentclass[bachelor, och, report]{../shiza}
% параметр - тип обучения - одно из значений:
%    spec     - специальность
%    bachelor - бакалавриат (по умолчанию)
%    master   - магистратура
% параметр - форма обучения - одно из значений:
%    och   - очное (по умолчанию)
%    zaoch - заочное
% параметр - тип работы - одно из значений:
%    referat    - реферат
%    coursework - курсовая работа (по умолчанию)
%    diploma    - дипломная работа
%    pract      - отчет по практике
% параметр - включение шрифта
%    times    - включение шрифта Times New Roman (если установлен)
%               по умолчанию выключен
\usepackage{subfigure}
\usepackage{tikz,pgfplots}
\pgfplotsset{compat=1.5}
\usepackage{float}

%\usepackage{titlesec}
\setcounter{secnumdepth}{4}
%\titleformat{\paragraph}
%{\normalfont\normalsize}{\theparagraph}{1em}{}
%\titlespacing*{\paragraph}
%{35.5pt}{3.25ex plus 1ex minus .2ex}{1.5ex plus .2ex}

\titleformat{\paragraph}[block]
{\hspace{1.25cm}\normalfont}
{\theparagraph}{1ex}{}
\titlespacing{\paragraph}
{0cm}{2ex plus 1ex minus .2ex}{.4ex plus.2ex}

% --------------------------------------------------------------------------%


\usepackage[T2A]{fontenc}
\usepackage[utf8]{inputenc}
\usepackage{graphicx}
\graphicspath{ {./images/} }
\usepackage{tempora}

\usepackage[sort,compress]{cite}
\usepackage{amsmath}
\usepackage{amssymb}
\usepackage{amsthm}
\usepackage{fancyvrb}
\usepackage{listings}
\usepackage{listingsutf8}
\usepackage{longtable}
\usepackage{array}
\usepackage[english,russian]{babel}

\usepackage[colorlinks=false]{hyperref}
\usepackage{url}

\usepackage{underscore}
\usepackage{setspace}
\usepackage{indentfirst} 
\usepackage{mathtools}
\usepackage{amsfonts}
\usepackage{enumitem}
\usepackage{tikz}
\usepackage{minted}

\newcommand{\eqdef}{\stackrel {\rm def}{=}}
\newcommand{\specialcell}[2][c]{%
\begin{tabular}[#1]{@{}c@{}}#2\end{tabular}}

\renewcommand\theFancyVerbLine{\small\arabic{FancyVerbLine}}

\newtheorem{lem}{Лемма}

\begin{document}

% Кафедра (в родительном падеже)
\chair{теоретических основ компьютерной безопасности и криптографии}

% Тема работы
\title{Практические задания по курсу ''Нейронные сети''}

% Курс
\course{5}

% Группа
\group{531}

% Факультет (в родительном падеже) (по умолчанию "факультета КНиИТ")
\department{факультета КНиИТ}

% Специальность/направление код - наименование
%\napravlenie{09.03.04 "--- Программная инженерия}
%\napravlenie{010500 "--- Математическое обеспечение и администрирование информационных систем}
%\napravlenie{230100 "--- Информатика и вычислительная техника}
%\napravlenie{231000 "--- Программная инженерия}
\napravlenie{100501 "--- Компьютерная безопасность}

% Для студентки. Для работы студента следующая команда не нужна.
% \studenttitle{Студентки}

% Фамилия, имя, отчество в родительном падеже
\author{Улитина Ивана Владимировича}

% Заведующий кафедрой
% \chtitle{} % степень, звание
% \chname{}

%Научный руководитель (для реферата преподаватель проверяющий работу)
\satitle{доцент} %должность, степень, звание
\saname{И. И. Слеповичев}

% Руководитель практики от организации (только для практики,
% для остальных типов работ не используется)
% \patitle{к.ф.-м.н.}
% \paname{С.~В.~Миронов}

% Семестр (только для практики, для остальных
% типов работ не используется)
%\term{8}

% Наименование практики (только для практики, для остальных
% типов работ не используется)
%\practtype{преддипломная}

% Продолжительность практики (количество недель) (только для практики,
% для остальных типов работ не используется)
%\duration{4}

% Даты начала и окончания практики (только для практики, для остальных
% типов работ не используется)
%\practStart{30.04.2019}
%\practFinish{27.05.2019}

% Год выполнения отчета
\date{2023}

\maketitle

% Включение нумерации рисунков, формул и таблиц по разделам
% (по умолчанию - нумерация сквозная)
% (допускается оба вида нумерации)
% \secNumbering

%-------------------------------------------------------------------------------------------

\tableofcontents

\section{Задание 1: Создание ориентированного графа}
    \subsection{Описание}

        \textbf{На входе:} текстовый файл с описанием графа в виде списка дуг:
        \[(a_1, b_1, n_1), (a_2, b_2, n_2), \dots, (a_k, b_k, n_k),\] где $a_i$
        "--- начальная вершина дуги $i$, $b_i$ "--- конечная вершина дуги $i$,
        $n_i$ "--- порядковый номер дуги в списке всех заходящих в вершину $b_i$
        дуг.

        \textbf{На выходе:} Ориентированный граф с именованными вершинами и
        линейно упорядоченными дугами (в соответствии с порядком из текстового
        файла). Сообщение об ошибке в формате файла, если ошибка присутствует.
        
        % \textbf{Реализация:}

    \subsection{Пример исполнения программы}

        Рассмотрим пример, созданный для программы в файле 'test1.txt', со
        следующим содержимым:

        \begin{minted}[breaklines,fontsize=\small]{text}
            (A, D, 1), (A, D, 2), (B, E, 1), (C, E, 2), (D, G, 1), (E, F, 1), (F, G, 2)
        \end{minted}

        Запускаем программу с помощью консоли следующим образом:

        \begin{minted}[breaklines,fontsize=\small]{text}
            python task1.py input=tests\task1\test1.txt output=tests\task1\task1_res.xml
        \end{minted}

        В качестве результата получаем файл 'task1_res.xml' с содержимым:

        \begin{minted}[breaklines,fontsize=\small]{text}
            <graph>
            <vertex>A</vertex>
            <vertex>D</vertex>
            <vertex>B</vertex>
            <vertex>E</vertex>
            <vertex>C</vertex>
            <vertex>G</vertex>
            <vertex>F</vertex>
            <arc>
              <from>A</from>
              <to>D</to>
              <order>1</order>
            </arc>
            <arc>
              <from>A</from>
              <to>D</to>
              <order>2</order>
            </arc>
            <arc>
              <from>D</from>
              <to>G</to>
              <order>1</order>
            </arc>
            <arc>
              <from>B</from>
              <to>E</to>
              <order>1</order>
            </arc>
            <arc>
              <from>E</from>
              <to>F</to>
              <order>1</order>
            </arc>
            <arc>
              <from>C</from>
              <to>E</to>
              <order>2</order>
            </arc>
            <arc>
              <from>F</from>
              <to>G</to>
              <order>2</order>
            </arc>
          </graph>          
        \end{minted}

\section{Задание 2: Создание функции по графу}
    \subsection{Описание}

        \textbf{На входе:} ориентированный граф с именованными вершинами как
        описано в задании 1.

        \textbf{На выходе:} линейное представление функции, реализуемой графом в
        префиксной скобочной записи: $$A_1(B_1(C_1(\dots), \dots, C_m(\dots)),
        \dots, B_n(\dots))$$

        % \textbf{Реализация:}

    \subsection{Пример исполнения программы}

        Рассмотрим пример, созданный для программы в файле 'test2.xml', со
        следующим содержимым:

        \begin{minted}[breaklines,fontsize=\small]{text}
            <graph>
            <vertex>A</vertex>
            <vertex>D</vertex>
            <vertex>B</vertex>
            <vertex>E</vertex>
            <vertex>C</vertex>
            <vertex>G</vertex>
            <vertex>F</vertex>
            <arc>
              <from>A</from>
              <to>D</to>
              <order>1</order>
            </arc>
            <arc>
              <from>A</from>
              <to>D</to>
              <order>2</order>
            </arc>
            <arc>
              <from>D</from>
              <to>G</to>
              <order>1</order>
            </arc>
            <arc>
              <from>B</from>
              <to>E</to>
              <order>1</order>
            </arc>
            <arc>
              <from>E</from>
              <to>F</to>
              <order>1</order>
            </arc>
            <arc>
              <from>C</from>
              <to>E</to>
              <order>2</order>
            </arc>
            <arc>
              <from>F</from>
              <to>G</to>
              <order>2</order>
            </arc>
          </graph>          
        \end{minted}

        Запускаем программу с помощью консоли следующим образом:

        \begin{minted}[breaklines,fontsize=\small]{text}
            python task2.py input=tests\task2\task2.xml output=tests\task2\task2_res.txt
        \end{minted}

        В качестве результата получаем файл 'task2_res.xml' с содержимым:

        \begin{minted}[breaklines,fontsize=\small]{text}
            G(D(A(), A()), F(E(B(), C()))) 
        \end{minted}

\section{Задание 3: Вычисление значение функции на графе}
    \subsection{Описание}

        \textbf{На входе:}
        \begin{enumerate}
            \item Текстовый файл с описанием графа в виде списка дуг (смотри задание 1).
            \item Текстовый файл соответствий арифметических операций именам вершин:
            
                \begin{center}
                    $a_1 : 1\text{-я операция}$ \\
                    $a_2 : 2\text{-я операция}$ \\
                    $\dots$ \\
                    $a_n : n\text{-я операция}$, \\
                \end{center}
                где $a_i$ -- имя $i$-й вершины, $i$-я операция -- символ операции, соответствующий вершине $a_i$.
                
                Допустимы следующие символы операций: \\
                $+$ -- cумма значений,\\
                $*$ -- произведение значений,\\
                $exp$ -- экспонирование входного значения,\\
                число -- любая числовая константа.\\		
        \end{enumerate}

        \textbf{На выходе:} значение функции, построенной по графу и файлу.
        
        % \textbf{Реализация:}

    \subsection{Пример исполнения программы}

        Рассмотрим пример, созданный для программы в файлах 'graph.txt' и
        'operations.txt', со следующим содержимым:

        \textbf{graph.txt}
        \begin{minted}[breaklines,fontsize=\small]{text}
            (v1, v4, 1), (v2, v4, 2), (v2, v5, 1), (v3, v5, 2), (v4, v6, 1), (v5, v6, 2), (v6, v7, 1)
        \end{minted}

        \textbf{operations.txt}
        \begin{minted}[breaklines,fontsize=\small]{text}
            {
                "v1" : 3,
                "v2" : 2,
                "v3" : 5,
                "v4" : "*",
                "v5" : "+",
                "v6" : "+",
                "v7" : "exp"
            }
        \end{minted}
        
        Запускаем программу с помощью консоли следующим образом:

        \begin{minted}[breaklines,fontsize=\small]{text}
            python task3.py graph=tests\task3\graph.txt ops=tests\task3\operations.txt output=tests\task3\res.txt
        \end{minted}

        В качестве результата получаем файл 'res.txt' с содержимым:

        \begin{minted}[breaklines,fontsize=\small]{text}
            442413.3920089205
        \end{minted}

\section{Задание 4: Построение многослойной нейронной сети}
    \subsection{Описание}

        \textbf{На входе:}
            \begin{enumerate}
                \item Файл с набором матриц весов межнейронных связей:
                \begin{center}
                    $M_1 : [a_{11}^1, a_{12}^1, \dots, a_{1n_1}^1], \dots, [a_{m_11}^1, a_{m_12}^1, \dots ,a_{m_1n_1}^1]$ \\
                    $M_2 : [a_{11}^2, a_{12}^2, \dots, a_{1n_2}^2], \dots, [a_{m_21}^2, a_{m_22}^2, \dots ,a_{m_2n_2}^2]$\\
                    $\dots$\\
                    $M_p : [a_{11}^p, a_{12}^p, \dots, a_{1n_p}^p], \dots, [a_{m_p1}^p, a_{m_p2}^p, \dots,a_{m_pn_p}^p]$ \\                  
                \end{center}
                \item Файл с входным вектором в формате:
                \begin{center}
                    $x_1, x_2, \dots, x_k$.
                \end{center}
            \end{enumerate}

        \textbf{На выходе:}
            \begin{enumerate}
                \item Сериализованная многослойная нейронная сеть с полносвязной
                межслойной структурой. Файл с выходным вектором -- результатом
                вычислений НС в формате: 
                \begin{center}
                    $y_1, y_2, \dots, y_k.$                
                \end{center}
                \item Сообщение об ошибке, если в формате входного вектора или
                файла описания НС допущена ошибка.
            \end{enumerate}

        % \textbf{Реализация:}

    \subsection{Пример исполнения программы}

        Рассмотрим пример, созданный для программы в файлах 'x4.txt' и 'w.txt',
        со следующим содержимым:

        \textbf{x4.txt}
        \begin{minted}[breaklines,fontsize=\small]{text}
            [2, 2, 8]
        \end{minted}
        
        \textbf{w.txt}
        \begin{minted}[breaklines,fontsize=\small]{text}
            [[[0.47519493033675375, 0.015705490366171526, 0.9433818257724572],
            [0.48092032736144574, 0.13929695479782134, 0.6869903232566065],
            [0.436988975888717, 0.20037642195993755, 0.17561406275527947]],
            [[0.042224071742743785, 0.15331022315027187, 0.464635658411239],
            [0.6000159964796773, 0.22606113281552231, 0.5301212736820182],
            [0.19651133783303198, 0.7498835958139106, 0.28721556978456597]],
            [[0.11837615025116721, 0.00927217999098906, 0.7504596929897048],
            [0.5675946231090779, 0.9748635791740536, 0.30501309542663524],
            [0.8574872089946126, 0.3047120321509168, 0.3376899733092712]]]
        \end{minted}

        Запускаем программу с помощью консоли следующим образом:

        \begin{minted}[breaklines,fontsize=\small]{text}
            task4.py x=tests\task4\x4.txt w=tests\task4\w.txt y=tests\task4\out.txt
        \end{minted}

        В качестве результата получаем файл 'out.txt' с содержимым:

        \begin{minted}[breaklines,fontsize=\small]{text}
            [0.6599800423450157, 0.7982164099813447, 0.7427805995966905]
        \end{minted}

\section{Задание 5: Реализация метода обратного распространения ошибки для многослойной НС}
    \subsection{Описание}

        \textbf{На входе:}
            \begin{enumerate}
                \item Текстовый файл с описанием НС (формат см. в задании 4).
                \item Текстовый файл с обучающей выборкой:
                    \begin{center}
                        $[x_{1}^{1}, x_{2}^1, \dots, x_{k}^1] \rightarrow [y_{1}^1, y_{2}^1, \dots, y_{l}^1]$ \\
                        $\dots$\\
                        $[x_{1}^n, x_{2}^n, \dots, x_{k}^n] \rightarrow [y_{1}^n, y_{2}^n, \dots, y_{l}^n]$ \\
                    \end{center}
                    Формат описания входного вектора $x$ и выходного вектора $y$ соответствует формату из задания 4. 
                \item Число итераций обучения (в строке параметров).
            \end{enumerate}

        \textbf{На выходе:} Текстовый файл с историей $N$ итераций обучения методом обратного распространения ошибки:
        \begin{center}
            $1 : \text{1-я ошибка}$ \\
            $2 : \text{2-я ошибка}$ \\
                $\dots$\\
            $N : \text{N-я ошибка}$ \\
        \end{center}

        % \textbf{Реализация:}

    \subsection{Пример исполнения программы}

        Рассмотрим пример, созданный для программы в файлах 'x5.txt', 'w.txt' и
        'y.txt', со следующим содержимым:

        \textbf{x5.txt}
        \begin{minted}[breaklines,fontsize=\small]{text}
            [
                [-4, 1, 5],
                [7, -1, -4],
                [4, 14, 10],
                [-8, -18, 6],
            ]
        \end{minted}

        \textbf{w.txt}
        \begin{minted}[breaklines,fontsize=\small]{text}
            [[[0.47519493033675375, 0.015705490366171526, 0.9433818257724572],
            [0.48092032736144574, 0.13929695479782134, 0.6869903232566065],
            [0.436988975888717, 0.20037642195993755, 0.17561406275527947]],
            [[0.042224071742743785, 0.15331022315027187, 0.464635658411239],
            [0.6000159964796773, 0.22606113281552231, 0.5301212736820182],
            [0.19651133783303198, 0.7498835958139106, 0.28721556978456597]],
            [[0.11837615025116721, 0.00927217999098906, 0.7504596929897048],
            [0.5675946231090779, 0.9748635791740536, 0.30501309542663524],
            [0.8574872089946126, 0.3047120321509168, 0.3376899733092712]]]
        \end{minted}

        \textbf{y.txt}
        \begin{minted}[breaklines,fontsize=\small]{text}
            [
                [0, 0, 0],
                [1, 1, 1],
                [1, 1, 1],
                [0, 0, 0],
            ]
        \end{minted}

        Запускаем программу с помощью консоли следующим образом:

        \begin{minted}[breaklines,fontsize=\small]{text}
            python task5.py x=tests\task5\x5.txt y=tests\task5\y.txt w=tests\task5\w.txt epochs=10 loss=tests\task5\results.txt
        \end{minted}

        В качестве результата получаем файл 'result.txt' с содержимым:

        \textbf{result.txt}
        \begin{minted}[breaklines,fontsize=\small]{text}
            Ошибка на эпохе 0 равна 0.2867640154236057
            Ошибка на эпохе 1 равна 0.2832965082336704
            Ошибка на эпохе 2 равна 0.28011549588401924
            Ошибка на эпохе 3 равна 0.27719508418488775
            Ошибка на эпохе 4 равна 0.2745690438749986
            Ошибка на эпохе 5 равна 0.2722617577309298
            Ошибка на эпохе 6 равна 0.2702545573996806
            Ошибка на эпохе 7 равна 0.26849787858212437
            Ошибка на эпохе 8 равна 0.2669367574826795
            Ошибка на эпохе 9 равна 0.265525581639051         
        \end{minted}

        % \begin{figure}[H]
        %     \centering
        %     \includegraphics[width=0.8\textwidth]{pic/3.png}
        %     \caption{Тест индекс-метода}
        % \end{figure}
\end{document}
